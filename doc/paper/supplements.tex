\PassOptionsToPackage{top=3cm,left=3cm,right=3cm,bottom=3cm}{geometry}
\documentclass[fleqn,11pt]{wlscirep_supp}

\usepackage[]{minitoc}
\mtcsetdepth{secttoc}{3}
\setcounter{secnumdepth}{2}
\setcounter{tocdepth}{2}
\mtcsettitle{secttoc}{}


\usepackage[utf8]{inputenc}
\usepackage[T1]{fontenc}
\usepackage[english]{babel}
%\usepackage[top=3cm,left=3cm,right=3cm,bottom=3cm]{geometry}% by courtesy of Mico
\usepackage{lmodern}
\usepackage{bbm}
\usepackage{graphicx}
\usepackage{epstopdf}
\usepackage{colortbl}
\usepackage{siunitx}
\sisetup{
  detect-all,
  detect-weight=true,
  detect-family=true,
  mode=text,
%   detect-inline-family=math,
  group-separator={,},
%   group-minimum-digits={3}			
}
\usepackage{rotating}
\usepackage{tabularx}
\usepackage{tabu}
\usepackage{authblk}
\usepackage{mathtools}
\usepackage{overpic}
\usepackage{url}
\usepackage{tikz}
\usetikzlibrary{positioning}
\usetikzlibrary{arrows}
\usetikzlibrary{fit}
\usepackage{multirow}
\usepackage{float}
\usepackage[normalem]{ulem}
\usepackage{bm}
\usepackage{enumerate}
\usepackage[absolute,overlay%,showboxes
                        ]{textpos}
% \usepackage{caption}
\usepackage[font=small,labelfont=bf,justification=justified]{caption}
\usepackage{subcaption}
\usepackage{xspace}
\usepackage[colorinlistoftodos]{todonotes}
\usepackage{placeins}
\usepackage{makecell, booktabs}
\usepackage{eqparbox}
\usepackage{rotating}
\usepackage{graphicx}
\usepackage{xspace}
\usepackage{setspace}
%\usepackage{comment}
\usepackage[resetlabels,labeled]{multibib}
\newcites{Supp}{References}
\usepackage[sort&compress]{cleveref}
\Crefname{appendix}{Supplement}{Supplements}

%\usepackage{mathabx}

% Table formatting packages
\usepackage{dcolumn} % align decimal points in tables
\newcolumntype{d}[1]{D{.}{.}{#1}}
\usepackage{booktabs} 
\usepackage[flushleft]{threeparttable}
\usepackage{siunitx} % align on decimal point in tables
\usepackage{lineno}
\usepackage{etoolbox}

\usepackage{lscape}
\usepackage{longtable}
\usepackage{arydshln}

% math
\usepackage{amsmath,amsfonts,amssymb}
% additional math symbols
\DeclareFontFamily{U}{mathb}{}
\DeclareFontShape{U}{mathb}{m}{n}{
  <-5.5> mathb5
  <5.5-6.5> mathb6
  <6.5-7.5> mathb7
  <7.5-8.5> mathb8
  <8.5-9.5> mathb9
  <9.5-11.5> mathb10
  <11.5-> mathb12
}{}
\DeclareSymbolFont{mathb}{U}{mathb}{m}{n}
\DeclareMathSymbol{\ulsh}{3}{mathb}{"E8}
\DeclareMathSymbol{\ursh}{3}{mathb}{"E9}
\DeclareMathSymbol{\dlsh}{3}{mathb}{"EA}
\DeclareMathSymbol{\drsh}{3}{mathb}{"EB}

%% Patch 'normal' math environments:
\newcommand*\linenomathpatch[1]{%
  \cspreto{#1}{\linenomath}%
  \cspreto{#1*}{\linenomath}%
  \csappto{end#1}{\endlinenomath}%
  \csappto{end#1*}{\endlinenomath}%
}

\linenomathpatch{equation}
\linenomathpatch{gather}
\linenomathpatch{multline}
\linenomathpatch{align}
\linenomathpatch{alignat}
\linenomathpatch{flalign}

\linenumbers

% PLOS formatting
\makeatletter %only needed in preamble
\renewcommand\Large{\@setfontsize\Large{18pt}{18}}
\renewcommand\large{\@setfontsize\large{16pt}{18}}
\makeatother

\addto\captionsenglish{\renewcommand{\figurename}{Figure}}

% \usepackage{xstring}
% \usepackage{etoolbox}
% \usepackage{caption}

% \captionsetup{labelfont=bf,tableposition=top}

% \makeatletter
% \newcommand\formatlabel[1]{%
%     \noexpandarg
%     \IfSubStr{#1}{.}{%
%       \StrBefore{#1}{.}[\firstcaption]%
%       \StrBehind{#1}{.}[\secondcaption]%
%       \textbf{\firstcaption.} \secondcaption}{%
%       #1}%
%       }


% \patchcmd{\@caption}{#3}{\formatlabel{#3}}
% \makeatother

\renewcommand*{\Affilfont}{\normalsize\normalfont}
\renewcommand*{\Authfont}{\normalfont}


% referencing of unnumbered materials and methods
\newcounter{methods}
\renewcommand{\themethods}{Materials and methods}

% Track changes
%\usepackage[markup=underlined]{changes}
\makeatletter
\@namedef{Changes@AuthorColor}{magenta}
\colorlet{Changes@Color}{magenta}
\makeatother


%=====================================================================% Declare

\DeclareSIUnit\eur{\officialeuro}
\DeclareSIUnit\M{M}
\DeclareSIUnit\k{k}

% Widebar symbol
% \DeclareFontFamily{U}{mathx}{\hyphenchar\font45}
% \DeclareFontShape{U}{mathx}{m}{n}{<-> mathx10}{}
% \DeclareSymbolFont{mathx}{U}{mathx}{m}{n}
% \DeclareMathAccent{\widebar}{0}{mathx}{"73}

%=====================================================================% New commands (Macros)

% def
\def\sym#1{\ifmmode^{#1}\else\(^{#1}\)\fi}
\definecolor{darkgreen}{rgb}{0.0, 0.5, 0.0}

% new command
\newcommand{\smallsim}{\smallsym{\mathrel}{\sim}}
\newcommand{\specialcell}[2][c]{%
  \begin{tabular}[#1]{@{}l@{}}#2\end{tabular}}
\newcommand{\specialcellc}[2][c]{%
  \begin{tabular}[#1]{@{}c@{}}#2\end{tabular}}
\newcommand\ie{i.\,e.\xspace}
\newcommand\eg{e.\,g.\xspace}
\newcommand{\dd}[1][]{\mathrm{d}#1}
\newcommand{\BK}[1]{{\color{orange}{BK: #1}}}
\newcommand{\figletter}[1]{{{\fontfamily{\sfdefault}\selectfont \textbf{#1}}}}
\newcommand\TODO[1]{{\color{red}#1}}  
\newcommand{\FIX}[1]{{\color{darkgreen}#1}}  

% renewcommand
\renewcommand\theadfont{\bfseries}
\renewcommand\theadalign{lc}
\renewcommand\cellalign{tl}

\makeatletter

\newbox\@abstract%
\def\abstitle{\textbf{Abstract}}%
\renewenvironment{abstract}{
  \global\setbox\@abstract\vbox\bgroup%
   \noindent
}{%
   \egroup%
}%

\renewcommand*{\Affilfont}{\normalsize\normalfont}
\renewcommand*{\Authfont}{\normalfont}

\addto\captionsenglish{% Replace "english" with the language you use
  \renewcommand{\contentsname}{List of Texts}
}

\def\@maketitle{%
  \newpage
    {\raggedright\fontsize{18pt}{20pt}\selectfont \@title \par}%
    \vskip 0.5em%
    {\large
      \lineskip .5em%
      \begin{tabular}[t]{l}%
        \raggedright \normalsize\mdseries{\@author} %
      \end{tabular}\par}%
      \vskip 1em
%      \raggedright\Large\abstitle\par
%      \vskip 1em
%    {\unvbox\@abstract\par}%
    \par
  \vskip 0.5em
}
  
\makeatother


\renewcommand{\thesection}{Text \arabic{section}}
\usepackage{titlesec}
%\titleformat{\section}{\normalfont\Large\bfseries}{Text \thesection.~#1}{1em}{}
\renewcommand{\thefigure}{S\arabic{figure}}
\renewcommand{\thetable}{S\arabic{table}}

\begin{document}
\doublespacing
\nolinenumbers

\newcommand{\supp}{SI Appendix}

\title{\LARGE\singlespacing{\textbf{S1 Appendix} \\ \medskip
Spatiotemporal modeling of \emph{Mtb} transmission in a South African primary care clinic}}

% author list
\author[1$\ddag$]{Nicolas Banholzer}
\author[2]{Keren Middelkoop}
\author[2]{Juane Leukes}
\author[1]{Kathrin Zürcher}
\author[2]{Robin Wood}
\author[1]{Matthias Egger}
\author[1*]{Lukas Fenner}

\affil[1]{Institute of Social and Preventive Medicine, University of Bern, Bern, Switzerland}
\affil[2]{Desmond Tutu HIV Centre, Department of Medicine, University of Cape Town, Cape Town, South Africa}

\affil[*]{Corresponding author: lukas.fenner@ispm.unibe.ch }

\affil[$\ddag$]{These authors contributed equally to this work.}

%\begin{abstract}\normalfont
%The supplementary material contains (1)~the detailed method, (2)~the simulation-based study, (3)~further descriptives, and (4)~the results from the sensitivity analysis.
%\end{abstract}

\flushbottom
\maketitle
\thispagestyle{empty}

%\newpage

\sloppy
\raggedbottom

\newpage

\appendix

\tableofcontents

\listoffigures

\listoftables

%\listoffigures
%\listoftables

\newpage

\section{Spatiotemporal model}\label{sec:spattemp-model}

\subsection{The Wells-Riley model}

We extend the Wells-Riley model [Wells-Riley], which is an established epidemiological model to estimate the risk of infection. The model estimates the probability of airborne indoor transmission $P$ using the following equation: 
\begin{align}
    P = \frac{C}{S} = 1 - \exp \left(-\frac{Ipqt}{Q}\right),
\end{align}
where $C$ is the number of disease cases, $S$ is the number of susceptibles, $I$ is the number of infectious individuals in space, $p$ is the breathing rate per person, $q$ is the generation rate of infectious quanta, $t$ is the duration of exposure, and $Q$ is the outdoor air supply rate. 

The unknown parameter $q$ is not the actual number of infectious particles in the air. It rather represents the dose of infectious particles that corresponds to a probability of infection with a Poisson relationship, \eg one quantum corresponds to $P = 1 - \exp (1) = 63\%$ risk of infection [Rudnick and Milton]. It is an attempt to consider the stochastic behavior of airborne transmission, which depends on the environment, characteristics of the pathogen, and characteristics of the individual. For example, the type of respiratory (\eg breathing, coughing, or sneezing) of the infectious individual influences the size distribution of the generated particles, with smaller particles getting more easily deposited in the lower respiratory tract of the susceptible person [Wang], increasing the probability of infection. 

While considering the stochastic behaviour of airborne transmission, the Wells-Riley model makes two simplifying assumptions. First, it assumes a well-mixed airspace, which means that the generated quanta disperse immediately and evenly in the indoor airspace. In other words, the exposure to infectious quanta is the same regardless of the locations of the infectious and susceptible individual. Second, the Wells-Riley model assumes steady-state conditions, which means that the quanta concentration and the outdoor air supply rate are constant over time. In other words, the risk of infection for each susceptible depends only on the duration of exposure (the time they spent in the indoor space). 

Rudnick and Milton [Rudnick and Milton] modified the Wells-Riley model to relax the steady-state assumption. They use CO$_2$ as a biomarker for exhaled breath to compute the outdoor air supply rate, which is otherwise difficult to measure. Their modified equation is
\begin{align}
    P = \frac{C}{S} = 1 - \exp \left(-\frac{\bar{f}Iqt}{n}\right),
\end{align}
where $f = \frac{CO^{\text{I}}-CO^{\text{O}}}{CO^{\text{A}}}$ is the fraction of indoor air that is exhaled breath, which is computed based on the CO$_2$ level in indoor air $CO^{\text{I}}$, outdoor air $CO^{\text{O}}$, and exhaled breath $CO^{\text{A}}$ (in parts per million [ppm]). 

While the extension by Rudnick and Milton resolves the steady-state assumption, it still assumes a well-mixed airspace. Therefore, our aim is to extend the Wells-Riley model further to allow the risk of infection to vary both across time and airspace. We accomplish this by modeling the quanta concentration spatiotemporally. The risk of infection then is the cumulative exposure to infectious quanta over time and space, which depends on the susceptible individual's time-varying positions in the indoor space and quanta concentration at these time points. Formally, we estimate the cumulative risk of infection as 
\begin{equation}
    P = \sum_s \sum_t N_{s,t} \cdot p,
\end{equation}
where $N$ is the quanta concentration in unit airspace $s$ at time $t$ (in quanta/$m^3$), and $p$ is the breathing rate per person (in $m^3/s$). The spatial quanta concentration changes over time according to the generation, diffusion and removal of quanta: 
\begin{itemize}
    \item[\ref{sec:quanta-generation}] Quanta generation: Infectious individuals generate quanta at their location in the indoor space. 
    \item[\ref{sec:quanta-diffusion}] Quanta diffusion: Over time, the infectious quanta diffuse and we approach a well-mixed airspace. 
    \item[\ref{sec:quanta-removal}] Quanta removal: Infectious quanta is removed from the air through (a)~outdoor air exchange, (b)~viral inactivation, and (c)~gravitational settling. 
\end{itemize}
We describe each process in more detail in the following. While we apply our model to \emph{Mtb}, we note that it can be applied to airborne respiratory infections in general.

\subsection{Setup}

We divide the airspace in a 2-dimensional grid, irrespective of the height of the airspace. The cell denotes the unit airspace with $s = 1, \dots, S$, where $(x_s, y_s)$ are the x and y coordinate of cell $s$. In some cases, we may use the (x,y)-notation in place of the cell ID $s$. The quanta concentration will be the same within each grid cell $s$, thus the grid provides a discrete approximation to the spatially varying quanta concentration. The number of grid cells $S = n_x \cdot n_y$ corresponds to the size of the matrix and determines the quality of the approximation as well as the model run time. More grid cells (smaller grid cells) provide a more granular picture of the spatial quanta concentration, at the expense of higher computation time. 

\subsection{Notation}

\begin{itemize}
    \item $N_{s,t}$: quanta concentration in airspace unit $s$ at time $t$
    \item $I_{s,t}$: number of infectious individuals in airspace unit $s$ at time $t$
    \item $q$: quanta generation rate (in quanta/s)
    \item $D$: diffusion constant (in m$^2$/s)
    \item $AER$: air change rate (in air changes/s)
    \item $V$: volume of the airspace (in m$^2$)
    \item $CO^{\text{I}}$: CO$_2$ concentration in indoor air
    \item $CO^{\text{O}}$: CO$_2$ concentration in outdoor air
    \item $G$: average CO$_2$ generation rate per person (in L$\cdot$min$^{-1}\cdot$person$^{-1}$)
    \item $n_t$: number of individuals in airspace at time $t$
    \item $\lambda$: viral inactivation rate (in quanta/s)
\end{itemize}

\subsection{Quanta generation}\label{sec:quanta-generation}

The number of quanta generated by infectious individuals at time $t$ is 
\begin{align}\label{eq:generation}
    I_t \cdot q ~.
\end{align}
The initial spatial spread of the quanta over the grid is informed prior literature. The concentration of virus-laden aerosols is highest closest to the infectious individual [References], in line with findings showing that the risk of infection correlates with the distance to the infectious individual [References]. Although coughing and sneezing can transport aerosols further away from the individual, these activities are much less frequent than breathing. Therefore, prior work suggests that breathing may contribute more infectious particles to the airspace than coughing or sneezing [Dinkele]. Overall, it seems plausible that the majority of infectious aerosols are generated near the infectious individual and then disperse over time. In line with prior findings, we will assume that virus-laden aerosols reach a distance of 0.50m to 1.0m through exhalation [References]. Since we do not observe the directional flow of the exhaled breadth, we assume that the generated quanta is evenly and radially distributed around the individual. 

\subsection{Quanta diffusion}\label{sec:quanta-diffusion}

Absent further knowledge on airflow, we assume that aerosols diffuse uniformly in the indoor air, over time approaching a well-mixed airspace. Initially, the quanta concentration will be highest at the location where it is generated and then disperse radially in (x,y)-direction towards locations where the concentration is lower. The diffusion equation is 
\begin{align}\label{eq:diffusion}
    \frac{\delta N_{s,t}}{\delta t} = D \Delta N_{s,t},
\end{align}
where $\Delta$ is the Laplace operator and $D$ is the diffusion constant (in m$^2$/s). We solve this equation with a standard second-order finite difference model using the Euler method with a 5-point stencil. 

The diffusion constant $D$ is unknown but can be approximated using an empirical relationship between the eddy diffusion coefficient $K$ and the outdoor air exchange rate $AER$ (in air changes/h)
\begin{align}
    K = (0.52 \cdot AER + 8.61e-5) * V^(2/3),
\end{align}
where $V$ is the volume of the airspace (in m$^3$). We assume that $D \approx K$. 

The $AER$ can be estimated from indoor CO$_2$ levels and room occupancy. Under steady-state conditions (CO$_2$ levels reaching a steady-state concentration), $AER$ can be computed as
\begin{align}
    AER = \frac{6e4 \cdot n \cdot G}{V\cdot(CO^{\text{I}}-CO^{\text{O}})},
\end{align}
where $n$ is the number of people in the airspace, $G$ is the average CO$_2$ generation rate per person (in L$\cdot$min$^{-1}\cdot$person$^{-1}$), $CO^{\text{I}}$ is the CO$_2$ concentration in indoor and $CO^{\text{O}}$ is the concentration in outdoor air (both in parts per million [ppm]) [Battermann]. In a primary care clinic, room occupancy varies continuously and thus it may be difficult to determine the steady-state CO$_2$ concentration. Therefore, we determine $AER$ using a transient mass balance model of the form [Battermann]
\begin{align}
    CO_{t+1}^{\text{I}} = \frac{6e4 \cdot n_t + G}{Q} \cdot \left(1 - \exp(-Q/V \Delta t)\right) + (CO_t^{\text{I}}-CO^{\text{O}}) \cdot \exp(-Q/V \Delta t) + CO^{\text{O}},
\end{align}
where $Q$ is the outdoor air supply rate (in m$^3$/h) and $AER = Q/V$. We numerically solve the transient mass balance model with the limited memory Broyden–Fletcher–Goldfarb–Shanno algorithm (L-BFGS) [REF], a quasi-Newton method. The optimal solution is the outdoor air supply rate $Q$ that minimizes the root mean square error. We also follow recommendations to fit the outdoor CO$_2$ level with reasonable constraints $CO^{\text{O}} = [350,500]$ [Battermann]. The $AER$ may be computed at different time intervals, thus we add the subscript $t$, \ie $AER_t$

\subsection{Quanta removal}\label{sec:quanta-removal}

Infectious quanta may be removed from the air through outdoor air exchange, viral inactivation, or gravitational settling. We will ignore gravitational settling. Vuorinen et al. [Reference] notes that particles smaller than 20\,$\mu$m rapidly become airborne. This is far above the upper end of the particle size distribution of most respiratory infections, including \emph{Mtb}, which is carried in particles typically smaller than 4.7\,$\mu$m [Fennelly]. The removal rate is thus the same of the outdoor air exchange rate $AER_t$ and the viral inactivation rate $\lambda$, \ie 
\begin{align}\label{eq:removal}
    AER_t + \lambda ~.
\end{align}
Note that $\lambda$ depends on the pathogen and is generally difficult to estimate and likely also depends on environmental factors, particularly relative humidity [Reference]. 

Considering the generation (\Cref{eq:generation}), diffusion (\Cref{eq:diffusion}), and removal (\Cref{eq:removal}) of quanta together, the quanta concentration at time $t$ in the airspace is computed as
\begin{align}
    \underbrace{N_{t}}_{\text{new concn.}} = \underbrace{\left(D \Delta (\underbrace{N_{t-1}}_{\text{prev. concn.}} + \underbrace{I_t \cdot q}_{\text{generation}})\right)}_{\text{diffusion}} \cdot \underbrace{\exp\left(-(AER_t + \lambda)\right)}_{\text{removal}} ~.
\end{align}

\subsection{Illustrative example}

We illustrate our modeling approach with an example. Consider a room with a volume $V$=150m$^3$ (length [L] = 10m, breadth [B] = 5m, height [H] = 3m). We divide the room into a grid of $S = 40 \cdot 20 = 800$ cubic cells with dimensions 0.25m$^2 \cdot 3$m, \ie a diagonal length of $\sqrt{0.25^2 + 0.25^2} = 0.35$\,m. We assume that quanta generation is confined to the cell where the infectious individual is located as well as the first neighbouring cells (\ie initially quanta reach a maximum distance of 0.7m through exhalation corresponding to the diagonal length of two cells). We monitor the room from 8am to 12am and consider one infectious individual generating quanta continuously for one hour between 8am and 9am at a fixed location in the middle of the room, \ie cell (20,10), so that the generation of quanta will be confined to cells $(20\pm1,10\pm1)$. We further assume a quanta generation rate of $q = 10$\,quanta/h, a constant outdoor air exchange rate of $AER = 1$\,air changes/h (corresponding to a removal rate of also 1\,quanta/h and a diffusion coefficient of 0.007m$^2$/s), and a viral inactivation rate of $\lambda = 0.5$\,quanta/h. 

\Cref{fig:toy-example} shows the quanta concentration at different time points following the generation of the first quanta (8:00:00am) and last quanta (9:00:00am). The quanta is generated at the location of the infectious individual. The quanta concentration remains high around this location until the individual ceases to generate more quanta at 9am. The quanta diffuses relatively quickly and at 10min past 9am the airspace is well-mixed with quanta. Over time, the removal through outdoor air exchange and viral inactivation brings the concentration in the room back towards 0. 

\begin{figure}
    \centering
    \includegraphics{}
    \caption{Quanta concentration at different time points following the generation of the first quanta (8:00:00am) and last quanta (9:00:00am) from an infectious individual that is placed in a fixed location in the middle of the room with dimensions L=10$\times$B=\5$\times$H=3m.}
    \label{fig:toy-example}
\end{figure}

\clearpage



\section{Case study}

We divide the clinical day into a morning (7am to 12am) and afternoon period (12am to 5pm) and estimate the $AER$ separately for each period, allowing it to vary by daytime.

Previous studies estimated $q$ for \emph{Mtb} or SARS-CoV-2 using the Wells-Riley model [REFs]. The range of estimates is large, suggesting that the infectiousness of \emph{Mtb} patients varies considerably. We model $q$ with a Student-t distribution with 1 degree of freedom $t_1(\mu = 2, \sigma = 2.5)$ that is truncated at 0 (lower) and 200 (upper). The distribution is shown in \Cref{fig:quanta-distribution} along with the estimates from the literature.

\begin{figure}
    \centering
    \includegraphics{}
    \caption{Caption}
    \label{fig:quanta-distribution}
\end{figure}


A schematic view of the clinic is shown in \Cref{fig:clinic}. We divide the clinic into three airspaces: the waiting room (l x b x h), the corridor (l x b x h), and the TB room (l x b x h). Each room is divided into a grid of cubic cells with a fixed cell length and breath of 0.25m. The volume of the grid cells thus depends on the height of each airspace. 

\begin{figure}
    \centering
    \includegraphics{}
    \caption{Caption}
    \label{fig:clinic}
\end{figure}

We compute the quanta concentration at $t = 1, \dots, T$, which is in seconds, corresponding to the frequency of the tracking data. We assume that the quanta concentration at $t=0$ is zero anywhere in the airspace, \ie $N_{s,0} = 0 \quad \forall s$. In other words, we assume that all quanta from the previous day has been removed from the air at the start of the following day. 

\clearpage

\bibliography{references.bib}

\end{document}